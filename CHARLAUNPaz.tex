% se define una variable condicional para cada formato de salida
% a conditional variable is defined for each output format
\newif\ifPDF%
\newif\ifBNPDF%
\newif\ifEPUB%
\newif\ifHTML%
\newif\ifODT%

 \PDFtrue
% \BNPDFtrue
% \EPUBtrue
% \HTMLtrue
% \ODTtrue

% agregamos las referencias
% we added the references
\addbibresource{./files/gbTeXbib-CHARLAUNPaz.bib}

% agregamos el glosario
% we added the glossary

\newglossaryentry{@glo203-unpaz}{
type = \acronymtype,
name         = {UNPAZ:},
description  = {Universidad Nacional de José Clemente Paz},
first        = {Universidad Nacional de José C. Paz (UNPAZ)},
text         = {UNPAZ},
}
\newglossaryentry{@glo204-libro}{
name         = {Libro:},
description  = {(del latín \emph{liber, libri}) es una obra impresa, manuscrita o pintada en una serie de hojas de papel, pergamino, vitela u otro material, unidas por un lado (es decir, encuadernadas) y protegidas con tapas, también llamadas cubiertas. Un libro puede tratar sobre cualquier tema. Según la definición de la Unesco,​ un libro debe poseer al menos veinticinco hojas (49 páginas), pues de veinticuatro hojas o menos sería un folleto; y de una hasta cuatro páginas se consideran hojas sueltas (en una o dos hojas)},
first        = {libro},
text         = {libro},
}

% los metadatos son opcionales en el PDF a diferencia del resto que es obligatorio y se cargan automáticamente
% the metadata is optional in the PDF, unlike the rest which is mandatory and loaded automatically
\ifODT
\usepackage[unicode,hyperindex=true]{hyperref}
\else
	\ifPDF
	% control de inconsistencias de principio y fin de linea
	% inconsistency control of start and end of line (homeoarchy)
	\usepackage[hyphenation,homeoarchy,draft,homeoarchywordcolor=yellow,homeoarchycharcolor=yellow]{impnattypo}
	\usepackage[allcolors=magenta,colorlinks]{hyperref}
	\usepackage{hyperxmp}
	
\Configure{META}{
<dc:title></dc:title>\Hnewline
<dc:alternative></dc:alternative>\Hnewline
<dc:subject></dc:subject>\Hnewline
<dc:keyword></dc:keyword>\Hnewline
<dc:created></dc:created>\Hnewline
<dc:modified></dc:modified>\Hnewline
<dc:creator></dc:creator>\Hnewline
<dc:type>book</dc:type>\Hnewline
<dc:language>es-ES</dc:language>\Hnewline
<dc:publisher></dc:publisher>\Hnewline
<dc:license>Copyright</dc:license>\Hnewline
<dc:rights>https://es.wikipedia.org/wiki/Copyright</dc:rights>\Hnewline
}
\makeatother
\begin{document}
\EndPreamble


	\else
		\ifBNPDF
		\usepackage[cam,width=18truecm,height=25.5truecm,center]{crop}
		\usepackage[hidelinks]{hyperref}
		\usepackage{hyperxmp}
		
\Configure{META}{
<dc:title></dc:title>\Hnewline
<dc:alternative></dc:alternative>\Hnewline
<dc:subject></dc:subject>\Hnewline
<dc:keyword></dc:keyword>\Hnewline
<dc:created></dc:created>\Hnewline
<dc:modified></dc:modified>\Hnewline
<dc:creator></dc:creator>\Hnewline
<dc:type>book</dc:type>\Hnewline
<dc:language>es-ES</dc:language>\Hnewline
<dc:publisher></dc:publisher>\Hnewline
<dc:license>Copyright</dc:license>\Hnewline
<dc:rights>https://es.wikipedia.org/wiki/Copyright</dc:rights>\Hnewline
}
\makeatother
\begin{document}
\EndPreamble


		\else
			\ifEPUB
			\usepackage[hyperindex=true,allcolors=magenta,colorlinks]{hyperref}
			\fi
		\fi
	\fi
\fi

\usepackage{lipsum} % Incluye el paquete lipsum
\begin{document}
\frontmatter

\ifEPUB%
	\ifdefined\HCode
	\phantomsection
	\addcontentsline{toc}{section}{Portada}
	\coverimage{./media/cover.png}
	\clearpage
	\fi
\fi

\ifPDF
% página 1
\newpage
\thispagestyle{empty}
{\textcolor{white}{.}}

% página 2
\newpage
\thispagestyle{empty}
{\textcolor{white}{.}}

% página 3
\newpage
\thispagestyle{empty}
{\textcolor{white}{.}}

\vspace{30mm}

\begin{center}
	\LARGE{Literatura y pensamiento}
\end{center}

% página 4
\newpage
\thispagestyle{empty}
{\textcolor{white}{.}}

% página 5
\newpage
\thispagestyle{empty}
\begin{center}%,draft
{\sc\large{andrés racket}}\\ %compiladoras
\end{center}

\vspace{30mm}

\begin{center}
\LARGE{Literatura y pensamiento}\\\vspace{10mm}

\Large{Caja de herramientas para el análisis literario}
\end{center}

\vfill

\begin{figure}[b]
\centering
\includegraphics[width=40mm]{./media/unpaz_logo.png}
\end{figure}

% página 6
\newpage
\thispagestyle{empty}
\begin{figure}[t]
\centering
\vspace{-10mm}
\includegraphics[width=15mm]{./media/logo-GNU.png}\\
\sc{colección morral de apuntes}
\end{figure}

\noindent Racket, Andrés \\
\noindent Literatura y pensamiento: caja de herramientas para el análisis literario. 1a ed. Buenos Aires: 2023.\\
\noindent 000 p.; 15.5x23 cm.\\
\noindent ISBN 978-987-8262-33-8 \\
\noindent 1. Literatura Argentina. 2. Pensamiento Nacional. 3. Estudios Literarios. I. Kusinsky, Darío,
pref. II. Título. \\
\noindent CDD A860 \\
\noindent Fecha de catalogación: 00/00/20xx \\
\noindent \textcopyright~2023, Andrés Racket \\
\noindent \textcopyright~2023, \\
%\noindent Imagen de tapa: .\\
\noindent Hecho el depósito que marca la ley 11.723\\
%\noindent Impreso en Argentina, tirada de esta edición: 000 ejemplares\\

\vfill

\noindent Licencia Creative Commons - Atribución - No Comercial (by-nc). Se permite la generación de obras derivadas siempre que no se haga con fines comerciales. Tampoco se puede utilizar la obra original con fines comerciales. Esta licencia no es una licencia libre. Algunos derechos reservados: \url{http://creativecommons.org/licenses/by-nc/4.0/deed.es}.

\else
	\ifBNPDF
	% página 1
\newpage
\thispagestyle{empty}
{\textcolor{white}{.}}

% página 2
\newpage
\thispagestyle{empty}
{\textcolor{white}{.}}

% página 3
\newpage
\thispagestyle{empty}
{\textcolor{white}{.}}

\vspace{30mm}

\begin{center}
	\LARGE{Literatura y pensamiento}
\end{center}

% página 4
\newpage
\thispagestyle{empty}
{\textcolor{white}{.}}

% página 5
\newpage
\thispagestyle{empty}
\begin{center}%,draft
{\sc\large{andrés racket}}\\ %compiladoras
\end{center}

\vspace{30mm}

\begin{center}
\LARGE{Literatura y pensamiento}\\\vspace{10mm}

\Large{Caja de herramientas para el análisis literario}
\end{center}

\vfill

\begin{figure}[b]
\centering
\includegraphics[width=40mm]{./media/unpaz_logo.png}
\end{figure}

% página 6
\newpage
\thispagestyle{empty}
\begin{figure}[t]
\centering
\vspace{-10mm}
\includegraphics[width=15mm]{./media/logo-GNU.png}\\
\sc{colección morral de apuntes}
\end{figure}

\noindent Racket, Andrés \\
\noindent Literatura y pensamiento: caja de herramientas para el análisis literario. 1a ed. Buenos Aires: 2023.\\
\noindent 000 p.; 15.5x23 cm.\\
\noindent ISBN 978-987-8262-33-8 \\
\noindent 1. Literatura Argentina. 2. Pensamiento Nacional. 3. Estudios Literarios. I. Kusinsky, Darío,
pref. II. Título. \\
\noindent CDD A860 \\
\noindent Fecha de catalogación: 00/00/20xx \\
\noindent \textcopyright~2023, Andrés Racket \\
\noindent \textcopyright~2023, \\
%\noindent Imagen de tapa: .\\
\noindent Hecho el depósito que marca la ley 11.723\\
%\noindent Impreso en Argentina, tirada de esta edición: 000 ejemplares\\

\vfill

\noindent Licencia Creative Commons - Atribución - No Comercial (by-nc). Se permite la generación de obras derivadas siempre que no se haga con fines comerciales. Tampoco se puede utilizar la obra original con fines comerciales. Esta licencia no es una licencia libre. Algunos derechos reservados: \url{http://creativecommons.org/licenses/by-nc/4.0/deed.es}.

	\fi
\fi

\tableofcontents

\chapter{Agradecimientos}
%\Author{Alberto Moyano}

Este \gls{@glo204-libro} comenzó a ser escrito en el año 2015, con el inicio del dictado de clases de Literatura y Pensamiento en las carreras de Producción y Desarrollo de Videojuegos y Producción y Gestión Audiovisual en la \gls{@glo203-unpaz}. A través de estos años hemos elaborado diversos programas, dia\-logado e interpretado textos en clase, conversado y polemizado entre colegas, y nos hemos encerrado para volver a estudiar todo lo que supuestamente sabíamos, hasta llegar a la forma actual de la unidad curricular. Este recorrido ha sido lo suficientemente amplio y enriquecedor como para que haya resultado interesante, finalmente, dejar registro de él en este libro. Sin la participación de los ya cientos de estudiantes que han cursado la unidad curricular y de los diversos colegas que han trabajado en ella, este libro no exis\-tiría. Es imposible, por lo tanto, no sentirse agradecido con todas estas personas y con la \gls{@glo203-unpaz} por un proceso de aprendizaje tan desafiante, fértil y nutritivo.

\chapter{Presentación}

En este \gls{@glo204-libro} resuenan voces de estudiantes y docentes que intercambiaron en el marco de la unidad curricular Literatura y Pensamiento los textos que aquí se manifiestan. Dicha unidad curricular pertenece a las carreras de Producción y Desarrollo de Videojuegos y Producción y Gestión Audiovisual del Departamento de Economía, Producción e Innovación Tecnológica (DEPIT) de la \gls{@glo203-unpaz}. Aquí se abordan un grupo de textos literarios que, según el autor, se pueden denominar \enquote{clásicos} de nuestra literatura nacional, escritos y publicados en su mayor parte durante el siglo~XX. Su lectura invita a un espacio de reflexión e intercambio en torno a la compleja y prolífica vinculación entre territorio y literatura. Los textos se posicionan, polemizan, reflexionan en torno a las disputas sociales, políticas e ideológicas de la época.

%La formación de \gls{@glo202-joseingenieros}\index[onomastico]{Ingenieros, José} fue universitaria y a diferencia del lugar tradicional del docente y del académico, su obra adquirió una marcada proyección política logrando complejas derivaciones que no son fácilmente encuadrables en un solo espacio partidario, si bien sus ideas estuvieron ligadas mayoritariamente a la tradición de izquierda socialista. Su vida y su obra habilitaron una diversidad de relecturas, de articulaciones políticas y de reapropiaciones teóricas. Esta particularidad caracterizó a \gls{@glo195-coas} y también a muchos de sus compañeros de militancia como el mexicano José Vasconcelos\index[onomastico]{Vasconcelos, José} o los argentinos Leopoldo Lugones\index[onomastico]{Lugones, Leopoldo} y Manuel Ugarte\index[onomastico]{Ugarte, Manuel} \parencite{@3070-TARKOVSKI1995}.

\chapter{Introducción}

\mainmatter

%cap1
\chapter{Fundaciones}

\section{\enquote{El Matadero} de Esteban Echeverría}

\section{La otra fundación: \emph{El gaucho Martín Fierro}}

\section{El encuentro entre Fierro y Cruz}

\section{Fundaciones}

\parencite{@3187-WELLS2014,@3188-SHELLY2023,@3189-QUIROGA2017,@3190-QUIROGA2016}
%cap2
\chapter{Metamorfosis}

\section{La batalla de las primeras personas}

\section{Los peligros del olvido de los orígenes}

%cap3
\chapter{La victoria, la muerte y la representación}

\section{Promulgación de la Ley N° 13010: discurso de Eva Duarte (1947)}

\section{Muerte y desaparición del cuerpo de Eva Duarte}

%cap4
\chapter{Viajes en el tiempo}

\lipsum[3-5]

\section{Argentinización de la galaxia: ciencia ficción}

\lipsum[4-9]

\section{Autobiografía: los viajes en el tiempo del realismo}

\lipsum[2-6]

\section{Ampliación de los límites y las fronteras de los géneros}

\lipsum[1-5]

\section{La literatura de terror: ¿para qué sirve narrar?}

\lipsum[1-5]

\section{Literatura contemporánea}

\lipsum[1-5]

%\section{SecciónA}
%
%Esta obra colectiva da continuidad a uno de los programas centrales impulsados por el \gls{@glo201-ceheal} desde su creación en 2019: el estudio de las ideas y del pensamiento económico en su vínculo con la implementación de políticas económicas.\footnote{Una nota a pie de página.}
%
%La secuencia de Fibonacci es una serie matemática en la cual cada número es la suma de los dos anteriores, comenzando generalmente con 0 y 1. La fórmula general para la secuencia de Fibonacci se puede expresar matemáticamente de la siguiente manera:
%
%$$F(n) = F(n-1) + F(n-2)$$
%
%\ifPDF
%\froufrou
%\else
%	\ifBNPDF
%	\froufrou
%	\else
%		\ifODT
%		\begin{center} * * * \end{center}
%		\fi
%	\fi
%\fi
%
%\section{SecciónAA}
%
%Esta obra colectiva da continuidad a uno de los programas centrales impulsados por el \gls{@glo201-ceheal} desde su creación en 2019: el estudio de las ideas y del pensamiento económico en su vínculo con la implementación de políticas económicas\index[concepto]{Políticas económicas para la sociedad} \parencite{@940-SHUMWAY1999}.
%
%\chapter{MainmatterB}
%
%\section{SecciónB}
%
%Esta obra colectiva da continuidad a uno de los programas centrales impulsados por el \gls{@glo201-ceheal} desde su creación en 2019: el estudio de las ideas y del pensamiento económico en su vínculo con la implementación de políticas económicas.
%
%\ifPDF%
%\begin{figure}[!ht]
%\centering
%\includegraphics[width=\textwidth]{./media/imagen1.jpg}
%\caption{Este es el epígrafe de la figura a color, véase \textcite{@3159-FUNES2006}.}
%\end{figure}
%\else
%	\ifBNPDF%
%	\begin{figure}[!ht]
%	\centering
%	\includegraphics[width=\textwidth]{./media/bn-imagen1.png}
%	\caption{Este es el epígrafe de la figura en escala de grises.}
%	\end{figure}
%	\else
%		\ifODT%
%		\begin{figure}[!ht]
%		\centering
%		\includegraphics[width=\textwidth]{./media/imagen1.jpg}
%		\caption{Este es el epígrafe de la figura a color.}
%		\end{figure}
%		\else
%			\ifEPUB
%			\begin{figure}[!ht]
%			\centering
%			\includegraphics[width=\textwidth]{./media/bn-imagen1.png}
%			\caption{Este es el epígrafe de la figura a color.}
%			\end{figure}
%			\fi
%		\fi
%	\fi
%\fi
%
%\section{SecciónBB}
%
%Esta obra colectiva da continuidad a uno de los programas centrales impulsados por el \gls{@glo201-ceheal} desde su creación en 2019: el estudio de las ideas y del pensamiento económico en su vínculo con la implementación de políticas económicas.\footnote{Nota a pie en una sección del segundo capítulo.}

\backmatter

%Condicional para llevar todas las notas a pie al final del libro como un capítulo.
%Conditional to move all the footnotes to the end of the book as a chapter
\ifEPUB
\begingroup
\parindent 0pt
\parskip 2ex
\def\enotesize{\normalsize}
\theendnotes
\endgroup
\fi

\ifPDF
\printnoidxglossary[type=\acronymtype,title={Índice de siglas}]
\printnoidxglossary[title={Glosario de términos}]
\printbibliography[heading=none,heading=bibintoc]
\else
	\ifBNPDF
	\printnoidxglossary[type=\acronymtype,title={Índice de siglas}]
	\printnoidxglossary[title={Glosario de términos}]
	\printbibliography[heading=none,heading=bibintoc]
	\else
		\ifODT
		\printnoidxglossary[type=\acronymtype,title={Índice de siglas}]
		\printnoidxglossary[title={Glosario de términos}]
		\printbibliography[heading=none,heading=bibintoc]
		\else
			\ifEPUB
			\printnoidxglossary[type=\acronymtype,title={Índice de siglas}]
			\printnoidxglossary[title={Glosario de términos}]
			\chapter{Referencias bibliográficas}
			\printbibliography[heading=none]
			\fi
		\fi
	\fi
\fi





\ifPDF
\printindex[names]
\printindex[concepto]
\printindex[onomastico]
\Author{Índice de autoras y autores}
\else
	\ifBNPDF
	\printindex[names]
	\printindex[concepto]
	\printindex[onomastico]
	\Author{Índice de autoras y autores}
% \else
% 	\ifEPUB
% 	\printindex[names]
% 	\printindex[concepto]
% 	\printindex[onomastico]
% 	\fi
	\fi
\fi

\chapter{Colofón}

La composición tipográfica de este libro se realizó utilizando gbTeXpublisher.

Las familias tipográficas utilizadas dentro del libro son: IBM Plex, una superfamilia de tipografía abierta, diseñada y desarrollada conceptualmente por Mike Abbink en IBM con colaboración de Bold Monday y Libertinus, bifurcación de la fuente Linux Libertine, diseñada para el texto del cuerpo y la lectura extendida.

\ifPDF
\newpage
\thispagestyle{empty}
{\textcolor{white}{.}}
\else
	\ifBNPDF
	\newpage
	\thispagestyle{empty}
	{\textcolor{white}{.}}
	\fi
\fi

\end{document}



