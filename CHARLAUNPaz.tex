% se define una variable condicional para cada formato de salida
\newif\ifPDF%
\newif\ifBLACKPDF%
\newif\ifHTML%
\newif\ifEPUB%
\newif\ifODT%

%\PDFtrue
%\BLACKPDFtrue
\HTMLtrue
%\EPUBtrue
%\ODTtrue

% agregamos las referencias
\addbibresource{./files/gbTeXbib-CHARLAUNPaz.bib}
% agregamos el glosario

\newglossaryentry{@glo203-unpaz}{
type = \acronymtype,
name         = {UNPAZ:},
description  = {Universidad Nacional de José Clemente Paz},
first        = {Universidad Nacional de José C. Paz (UNPAZ)},
text         = {UNPAZ},
}
\newglossaryentry{@glo204-libro}{
name         = {Libro:},
description  = {(del latín \emph{liber, libri}) es una obra impresa, manuscrita o pintada en una serie de hojas de papel, pergamino, vitela u otro material, unidas por un lado (es decir, encuadernadas) y protegidas con tapas, también llamadas cubiertas. Un libro puede tratar sobre cualquier tema. Según la definición de la Unesco,​ un libro debe poseer al menos veinticinco hojas (49 páginas), pues de veinticuatro hojas o menos sería un folleto; y de una hasta cuatro páginas se consideran hojas sueltas (en una o dos hojas)},
first        = {libro},
text         = {libro},
}

\ifPDF
\usepackage[hyphenation,homeoarchy,draft,homeoarchywordcolor=yellow,homeoarchycharcolor=orange]{impnattypo}
\usepackage[allcolors=magenta,colorlinks]{hyperref}

\Configure{META}{
<dc:title></dc:title>\Hnewline
<dc:alternative></dc:alternative>\Hnewline
<dc:subject></dc:subject>\Hnewline
<dc:keyword></dc:keyword>\Hnewline
<dc:created></dc:created>\Hnewline
<dc:modified></dc:modified>\Hnewline
<dc:creator></dc:creator>\Hnewline
<dc:type>book</dc:type>\Hnewline
<dc:language>es-ES</dc:language>\Hnewline
<dc:publisher></dc:publisher>\Hnewline
<dc:license>Copyright</dc:license>\Hnewline
<dc:rights>https://es.wikipedia.org/wiki/Copyright</dc:rights>\Hnewline
}
\makeatother
\begin{document}
\EndPreamble


\usepackage{hyperxmp}
	\else
	\ifBLACKPDF
	\usepackage[cam,width=18truecm,height=25.5truecm,center]{crop}
	\usepackage[draft]{hyperref}
	
\Configure{META}{
<dc:title></dc:title>\Hnewline
<dc:alternative></dc:alternative>\Hnewline
<dc:subject></dc:subject>\Hnewline
<dc:keyword></dc:keyword>\Hnewline
<dc:created></dc:created>\Hnewline
<dc:modified></dc:modified>\Hnewline
<dc:creator></dc:creator>\Hnewline
<dc:type>book</dc:type>\Hnewline
<dc:language>es-ES</dc:language>\Hnewline
<dc:publisher></dc:publisher>\Hnewline
<dc:license>Copyright</dc:license>\Hnewline
<dc:rights>https://es.wikipedia.org/wiki/Copyright</dc:rights>\Hnewline
}
\makeatother
\begin{document}
\EndPreamble


	\usepackage{hyperxmp}
		\else
		\ifEPUB
		\usepackage[hyperindex=true,colorlinks]{hyperref}
		
\Configure{META}{
<dc:title></dc:title>\Hnewline
<dc:alternative></dc:alternative>\Hnewline
<dc:subject></dc:subject>\Hnewline
<dc:keyword></dc:keyword>\Hnewline
<dc:created></dc:created>\Hnewline
<dc:modified></dc:modified>\Hnewline
<dc:creator></dc:creator>\Hnewline
<dc:type>book</dc:type>\Hnewline
<dc:language>es-ES</dc:language>\Hnewline
<dc:publisher></dc:publisher>\Hnewline
<dc:license>Copyright</dc:license>\Hnewline
<dc:rights>https://es.wikipedia.org/wiki/Copyright</dc:rights>\Hnewline
}
\makeatother
\begin{document}
\EndPreamble


			\else
			\ifHTML
			\usepackage{hyperref}
			
\Configure{META}{
<dc:title></dc:title>\Hnewline
<dc:alternative></dc:alternative>\Hnewline
<dc:subject></dc:subject>\Hnewline
<dc:keyword></dc:keyword>\Hnewline
<dc:created></dc:created>\Hnewline
<dc:modified></dc:modified>\Hnewline
<dc:creator></dc:creator>\Hnewline
<dc:type>book</dc:type>\Hnewline
<dc:language>es-ES</dc:language>\Hnewline
<dc:publisher></dc:publisher>\Hnewline
<dc:license>Copyright</dc:license>\Hnewline
<dc:rights>https://es.wikipedia.org/wiki/Copyright</dc:rights>\Hnewline
}
\makeatother
\begin{document}
\EndPreamble


			\title{Literatura y pensamiento}
			\date{\today}
				\else
				\ifODT
				\usepackage[draft]{hyperref}
%				se deshabilitan en gbTeXpublisher los metadatos para word
%				
\Configure{META}{
<dc:title></dc:title>\Hnewline
<dc:alternative></dc:alternative>\Hnewline
<dc:subject></dc:subject>\Hnewline
<dc:keyword></dc:keyword>\Hnewline
<dc:created></dc:created>\Hnewline
<dc:modified></dc:modified>\Hnewline
<dc:creator></dc:creator>\Hnewline
<dc:type>book</dc:type>\Hnewline
<dc:language>es-ES</dc:language>\Hnewline
<dc:publisher></dc:publisher>\Hnewline
<dc:license>Copyright</dc:license>\Hnewline
<dc:rights>https://es.wikipedia.org/wiki/Copyright</dc:rights>\Hnewline
}
\makeatother
\begin{document}
\EndPreamble


				\fi
			\fi
		\fi
	\fi
\fi

\begin{document}
\frontmatter

\ifEPUB
	\ifdefined\HCode
	\phantomsection
	\addcontentsline{toc}{section}{Portada}
	\coverimage{./media/cover.png}
	\clearpage
	\fi
\fi

\ifPDF
% página 1
\newpage
\thispagestyle{empty}
{\textcolor{white}{.}}

% página 2
\newpage
\thispagestyle{empty}
{\textcolor{white}{.}}

% página 3
\newpage
\thispagestyle{empty}
{\textcolor{white}{.}}

\vspace{30mm}

\begin{center}
	\LARGE{Literatura y pensamiento}
\end{center}

% página 4
\newpage
\thispagestyle{empty}
{\textcolor{white}{.}}

% página 5
\newpage
\thispagestyle{empty}
\begin{center}%,draft
{\sc\large{andrés racket}}\\ %compiladoras
\end{center}

\vspace{30mm}

\begin{center}
\LARGE{Literatura y pensamiento}\\\vspace{10mm}

\Large{Caja de herramientas para el análisis literario}
\end{center}

\vfill

\begin{figure}[b]
\centering
\includegraphics[width=40mm]{./media/unpaz_logo.png}
\end{figure}

% página 6
\newpage
\thispagestyle{empty}
\begin{figure}[t]
\centering
\vspace{-10mm}
\includegraphics[width=15mm]{./media/logo-GNU.png}\\
\sc{colección morral de apuntes}
\end{figure}

\noindent Racket, Andrés \\
\noindent Literatura y pensamiento: caja de herramientas para el análisis literario. 1a ed. Buenos Aires: 2023.\\
\noindent 000 p.; 15.5x23 cm.\\
\noindent ISBN 978-987-8262-33-8 \\
\noindent 1. Literatura Argentina. 2. Pensamiento Nacional. 3. Estudios Literarios. I. Kusinsky, Darío,
pref. II. Título. \\
\noindent CDD A860 \\
\noindent Fecha de catalogación: 00/00/20xx \\
\noindent \textcopyright~2023, Andrés Racket \\
\noindent \textcopyright~2023, \\
%\noindent Imagen de tapa: .\\
\noindent Hecho el depósito que marca la ley 11.723\\
%\noindent Impreso en Argentina, tirada de esta edición: 000 ejemplares\\

\vfill

\noindent Licencia Creative Commons - Atribución - No Comercial (by-nc). Se permite la generación de obras derivadas siempre que no se haga con fines comerciales. Tampoco se puede utilizar la obra original con fines comerciales. Esta licencia no es una licencia libre. Algunos derechos reservados: \url{http://creativecommons.org/licenses/by-nc/4.0/deed.es}.

	\else
	\ifBLACKPDF
	% página 1
\newpage
\thispagestyle{empty}
{\textcolor{white}{.}}

% página 2
\newpage
\thispagestyle{empty}
{\textcolor{white}{.}}

% página 3
\newpage
\thispagestyle{empty}
{\textcolor{white}{.}}

\vspace{30mm}

\begin{center}
	\LARGE{Literatura y pensamiento}
\end{center}

% página 4
\newpage
\thispagestyle{empty}
{\textcolor{white}{.}}

% página 5
\newpage
\thispagestyle{empty}
\begin{center}%,draft
{\sc\large{andrés racket}}\\ %compiladoras
\end{center}

\vspace{30mm}

\begin{center}
\LARGE{Literatura y pensamiento}\\\vspace{10mm}

\Large{Caja de herramientas para el análisis literario}
\end{center}

\vfill

\begin{figure}[b]
\centering
\includegraphics[width=40mm]{./media/unpaz_logo.png}
\end{figure}

% página 6
\newpage
\thispagestyle{empty}
\begin{figure}[t]
\centering
\vspace{-10mm}
\includegraphics[width=15mm]{./media/logo-GNU.png}\\
\sc{colección morral de apuntes}
\end{figure}

\noindent Racket, Andrés \\
\noindent Literatura y pensamiento: caja de herramientas para el análisis literario. 1a ed. Buenos Aires: 2023.\\
\noindent 000 p.; 15.5x23 cm.\\
\noindent ISBN 978-987-8262-33-8 \\
\noindent 1. Literatura Argentina. 2. Pensamiento Nacional. 3. Estudios Literarios. I. Kusinsky, Darío,
pref. II. Título. \\
\noindent CDD A860 \\
\noindent Fecha de catalogación: 00/00/20xx \\
\noindent \textcopyright~2023, Andrés Racket \\
\noindent \textcopyright~2023, \\
%\noindent Imagen de tapa: .\\
\noindent Hecho el depósito que marca la ley 11.723\\
%\noindent Impreso en Argentina, tirada de esta edición: 000 ejemplares\\

\vfill

\noindent Licencia Creative Commons - Atribución - No Comercial (by-nc). Se permite la generación de obras derivadas siempre que no se haga con fines comerciales. Tampoco se puede utilizar la obra original con fines comerciales. Esta licencia no es una licencia libre. Algunos derechos reservados: \url{http://creativecommons.org/licenses/by-nc/4.0/deed.es}.

	\fi
\fi

\tableofcontents

\ifHTML
\chapter{Legales}
Acá va con un nuevo tipo de configuración todos los elementos que hacen al área de legales del libro.
\fi

\chapter{Agradecimientos}
\ifPDF
\Author{Andrés Racket}
	\else
	\ifBLACKPDF
	\Author{Andrés Racket}
	\fi
\fi

Este \gls{@glo204-libro} comenzó a ser escrito en el año 2015, con el inicio del dictado de clases de Literatura y Pensamiento en las carreras de Producción y Desarrollo de Videojuegos y Producción y Gestión Audiovisual en la \gls{@glo203-unpaz}. A través de estos años hemos elaborado diversos programas, dia\-logado e interpretado textos en clase, conversado y polemizado entre colegas, y nos hemos encerrado para volver a estudiar todo lo que supuestamente sabíamos, hasta llegar a la forma actual de la unidad curricular. Este recorrido ha sido lo suficientemente amplio y enriquecedor como para que haya resultado interesante, finalmente, dejar registro de él en este libro. Sin la participación de los ya cientos de estudiantes que han cursado la unidad curricular y de los diversos colegas que han trabajado en ella, este libro no exis\-tiría. Es imposible, por lo tanto, no sentirse agradecido con todas estas personas y con la \gls{@glo203-unpaz} por un proceso de aprendizaje tan desafiante, fértil y nutritivo \parencite{@3187-WELLS2014}.

\chapter{Presentación}
\ifPDF
\Author{Darío Kusinsky}
	\else
	\ifBLACKPDF
	\Author{Darío Kusinsky}
	\fi
\fi

En este \gls{@glo204-libro} resuenan voces de estudiantes y docentes que intercambiaron en el marco de la unidad curricular Literatura y Pensamiento los textos que aquí se manifiestan. Dicha unidad curricular pertenece a las carreras de Producción y Desarrollo de Videojuegos y Producción y Gestión Audiovisual del Departamento de Economía, Producción e Innovación Tecnológica (DEPIT) de la \gls{@glo203-unpaz}. Aquí se abordan un grupo de textos literarios que, según el autor, se pueden denominar \enquote{clásicos} de nuestra literatura nacional, escritos y publicados en su mayor parte durante el siglo~XX. Su lectura invita a un espacio de reflexión e intercambio en torno a la compleja y prolífica vinculación entre territorio y literatura. Los textos se posicionan, polemizan, reflexionan en torno a las disputas sociales, políticas e ideológicas de la época \parencite{@3187-WELLS2014}.

\chapter{Introducción}
\ifPDF
\Author{Andrés Racket}
	\else
	\ifBLACKPDF
	\Author{Andrés Racket}
	\fi
\fi

Este libro elabora un recorrido a través de varias series de textos, todos ellos pertenecientes a nuestra literatura nacional. En su mayor parte, fueron escritos y publicados durante el siglo~XX, aunque los dos primeros son del siglo~XIX y los dos últimos del siglo~XXI. En la medida en que este recorrido es el que realizamos en el aula cuando dictamos la materia, constituye una introducción y puede ser, por lo tanto, ampliado, desarrollado y enriquecido.

Apostamos a que nuestros estudiantes y lectores continúen, a través de sus propias búsquedas e inquietudes, lo que se abre y se inicia en estas páginas. Durante el siglo~XX en nuestro país se produjo la irrupción del peronismo en la política nacional. La fecha que se suele elegir para marcar el inicio de esta etapa es el 17 de octubre de 1945. Durante la segunda mitad de ese siglo~XX y estas primeras décadas del siglo~XXI la vida política nacional prácticamente ha girado por completo en torno a este movimiento político. En su nacimiento, el peronismo defiende las banderas de la inclusión social y la ampliación de derechos de los trabajadores; es un movimiento que vehiculiza la llegada al poder y a las instituciones de sectores populares de nuestra sociedad que no habían contado, hasta ese momento, con representación política. Ejemplo de lista numerada:

\begin{enumerate}
	\item primer item;
	\item segundo item;
	\item tercer item.
\end{enumerate}

Las tensiones que provoca esta irrupción se reflejan en nuestra literatura: los textos se posicionan, polemizan, reflexionan y, en líneas generales, se hacen partícipes en el terreno de las representaciones de las disputas sociales, políticas e ideológicas de la época. En esa segunda mitad del siglo~XX, en relación directa con la aparición de este nuevo movimiento, se producen en nuestro país constantes sucesos de violencia: el bombardeo a Plaza de Mayo de 1955 y las dictaduras cívico-militares de 1955 y 1976 son los ejemplos más evidentes. La tradición literaria nacional no permanece ajena a esa violencia, ya sea para reproducirla en el plano simbólico o para denunciarla y luchar contra ella

Para comprender en forma precisa el vínculo entre literatura y política en ese período, o bien el vínculo entre literatura y violencia política, hay que ir, sin embargo, más atrás en el tiempo, en búsqueda de la fundación de nuestra tradición literaria nacional, pues desde sus inicios nuestra tradición literaria se constituye en derredor de una perspectiva fuertemente despreciativa y desdeñosa de lo popular. La irrupción del peronismo viene entonces a dar nuevos bríos y reconfigurar esa relación agresiva que nuestro mundo intelectual nacional había planteado, en realidad, mucho tiempo antes.

\ifPDF
\begin{figure}[!ht]
\centering
\includegraphics[width=\linewidth]{./media/imagen1.jpg}
\caption{Este es el epígrafe de la figura a color.}
\end{figure}
	\else
	\ifBLACKPDF
	\begin{figure}[!ht]
	\centering
	\includegraphics[width=\linewidth]{./media/bn-imagen1.png}
	\caption{Este es el epígrafe de la figura en escala de gris.}
	\end{figure}
		\else
		\ifHTML
		\begin{figure}
		\centering
		\includegraphics[width=\linewidth]{./media/imagen1.jpg}
		\caption{Este es el epígrafe de la figura a color.}
		\end{figure}
			\else
			\ifEPUB
			\begin{figure}
			\centering
			\includegraphics[width=\linewidth]{./media/imagen1.jpg}
			\caption{Este es el epígrafe de la figura a color.}
			\end{figure}
				\else
				\ifODT
				\begin{figure}
				\centering
				\includegraphics[width=\linewidth]{./media/image14.jpeg}
				\caption{Este es el epígrafe de la figura a color.}
				\end{figure}
				\fi
			\fi
		\fi
	\fi
\fi

Este libro elabora un recorrido a través de varias series de textos, todos ellos pertenecientes a nuestra literatura nacional. En su mayor parte, fueron escritos y publicados durante el siglo~XX, aunque los dos primeros son del siglo~XIX y los dos últimos del siglo~XXI. En la medida en que este recorrido es el que realizamos en el aula cuando dictamos la materia, constituye una introducción y puede ser, por lo tanto, ampliado, desarrollado y enriquecido.

Apostamos a que nuestros estudiantes y lectores continúen, a través de sus propias búsquedas e inquietudes, lo que se abre y se inicia en estas páginas. Durante el siglo~XX en nuestro país se produjo la irrupción del peronismo en la política nacional. La fecha que se suele elegir para marcar el inicio de esta etapa es el 17 de octubre de 1945. Durante la segunda mitad de ese siglo~XX y estas primeras décadas del siglo~XXI la vida política nacional prácticamente ha girado por completo en torno a este movimiento político. En su nacimiento, el peronismo defiende las banderas de la inclusión social y la ampliación de derechos de los trabajadores; es un movimiento que vehiculiza la llegada al poder y a las instituciones de sectores populares de nuestra sociedad que no habían contado, hasta ese momento, con representación política.

Las tensiones que provoca esta irrupción se reflejan en nuestra literatura: los textos se posicionan, polemizan, reflexionan y, en líneas generales, se hacen partícipes en el terreno de las representaciones de las disputas sociales, políticas e ideológicas de la época. En esa segunda mitad del siglo~XX, en relación directa con la aparición de este nuevo movimiento, se producen en nuestro país constantes sucesos de violencia: el bombardeo a Plaza de Mayo de 1955 y las dictaduras cívico-militares de 1955 y 1976 son los ejemplos más evidentes. La tradición literaria nacional no permanece ajena a esa violencia, ya sea para reproducirla en el plano simbólico o para denunciarla y luchar contra ella.

Para comprender en forma precisa el vínculo entre literatura y política en ese período, o bien el vínculo entre literatura y violencia política, hay que ir, sin embargo, más atrás en el tiempo, en búsqueda de la fundación de nuestra tradición literaria nacional, pues desde sus inicios nuestra tradición literaria se constituye en derredor de una perspectiva fuertemente despreciativa y desdeñosa de lo popular. La irrupción del peronismo viene entonces a dar nuevos bríos y reconfigurar esa relación agresiva que nuestro mundo intelectual nacional había planteado, en realidad, mucho tiempo antes.

Este libro elabora un recorrido a través de varias series de textos, todos ellos pertenecientes a nuestra literatura nacional. En su mayor parte, fueron escritos y publicados durante el siglo~XX, aunque los dos primeros son del siglo~XIX y los dos últimos del siglo~XXI. En la medida en que este recorrido es el que realizamos en el aula cuando dictamos la materia, constituye una introducción y puede ser, por lo tanto, ampliado, desarrollado y enriquecido.

Apostamos a que nuestros estudiantes y lectores continúen, a través de sus propias búsquedas e inquietudes, lo que se abre y se inicia en estas páginas. Durante el siglo~XX en nuestro país se produjo la irrupción del peronismo en la política nacional. La fecha que se suele elegir para marcar el inicio de esta etapa es el 17 de octubre de 1945. Durante la segunda mitad de ese siglo~XX y estas primeras décadas del siglo~XXI la vida política nacional prácticamente ha girado por completo en torno a este movimiento político. En su nacimiento, el peronismo defiende las banderas de la inclusión social y la ampliación de derechos de los trabajadores; es un movimiento que vehiculiza la llegada al poder y a las instituciones de sectores populares de nuestra sociedad que no habían contado, hasta ese momento, con representación política.

Las tensiones que provoca esta irrupción se reflejan en nuestra literatura: los textos se posicionan, polemizan, reflexionan y, en líneas generales, se hacen partícipes en el terreno de las representaciones de las disputas sociales, políticas e ideológicas de la época. En esa segunda mitad del siglo~XX, en relación directa con la aparición de este nuevo movimiento, se producen en nuestro país constantes sucesos de violencia: el bombardeo a Plaza de Mayo de 1955 y las dictaduras cívico-militares de 1955 y 1976 son los ejemplos más evidentes. La tradición literaria nacional no permanece ajena a esa violencia, ya sea para reproducirla en el plano simbólico o para denunciarla y luchar contra ella.

Para comprender en forma precisa el vínculo entre literatura y política en ese período, o bien el vínculo entre literatura y violencia política, hay que ir, sin embargo, más atrás en el tiempo, en búsqueda de la fundación de nuestra tradición literaria nacional, pues desde sus inicios nuestra tradición literaria se constituye en derredor de una perspectiva fuertemente despreciativa y desdeñosa de lo popular. La irrupción del peronismo viene entonces a dar nuevos bríos y reconfigurar esa relación agresiva que nuestro mundo intelectual nacional había planteado, en realidad, mucho tiempo antes.

\mainmatter

%cap1
\chapter{Fundaciones}

Quisque ullamcorper placerat ipsum. Cras nibh. Morbi vel justo vitae lacus tincidunt ultrices. Lorem ipsum dolor sit amet, consectetuer adipiscing elit. In hac habitasse platea dictumst. Integer tempus convallis augue. Etiam facilisis. Nunc elementum fermentum wisi. Aenean placerat. Ut imperdiet, enim sed gravida sollicitudin, felis odio placerat quam, ac pulvinar elit purus eget enim. Nunc vitae tortor. Proin tempus nibh sit amet nisl. Vivamus quis tortor vitae risus porta vehicula.

Fusce mauris. Vestibulum luctus nibh at lectus. Sed bibendum, nulla a faucibus semper, leo velit ultricies tellus, ac venenatis arcu wisi vel nisl. Vestibulum diam. Aliquam pellentesque, augue quis sagittis posuere, turpis lacus congue quam, in hendrerit risus eros eget felis. Maecenas eget erat in sapien mattis porttitor. Vestibulum porttitor.

\begin{quote}
	\enquote{Suspendisse vel felis. Ut lorem lorem, interdum eu, tincidunt sit amet, laoreet vitae, arcu. Aenean faucibus pede eu ante. Praesent enim elit, rutrum at, molestie non, nonummy vel, nisl} \parencite{@3188-SHELLY2023}.
\end{quote}

Nulla facilisi. Sed a turpis eu lacus commodo facilisis. Morbi fringilla, wisi in dignissim interdum, justo lectus sagittis dui, et vehicula libero dui cursus dui. Mauris tempor ligula sed lacus. Duis cursus enim ut augue. Cras ac magna. Cras nulla. Nulla egestas. Curabitur a leo. Quisque egestas wisi eget nunc. Nam feugiat lacus vel est. Curabitur consectetuer \parencite{@3190-QUIROGA2016}.

\section{\enquote{El Matadero} de Esteban Echeverría}

Fusce mauris. Vestibulum luctus nibh at lectus. Sed bibendum, nulla a faucibus semper, leo velit ultricies tellus, ac venenatis arcu wisi vel nisl. Vestibulum diam. Aliquam pellentesque, augue quis sagittis posuere, turpis lacus congue quam, in hendrerit risus eros eget felis. Maecenas eget erat in sapien mattis porttitor. Vestibulum porttitor \parencite{@3190-QUIROGA2016}.\footnote{Fusce mauris. Vestibulum luctus nibh at lectus. Sed bibendum, nulla a faucibus semper.}

\section{La otra fundación: \emph{El gaucho Martín Fierro}}

Fusce mauris. Vestibulum luctus nibh at lectus. Sed bibendum, nulla a faucibus semper, leo velit ultricies tellus, ac venenatis arcu wisi vel nisl. Vestibulum diam. Aliquam pellentesque, augue quis sagittis posuere, turpis lacus congue quam, in hendrerit risus eros eget felis \parencite{@3188-SHELLY2023}.\footnote{Fusce mauris. Vestibulum luctus nibh at lectus. Sed bibendum, nulla a faucibus semper, leo velit ultricies tellus, ac venenatis arcu wisi vel nisl. Vestibulum diam.}

\section{El encuentro entre Fierro y Cruz}

Fusce mauris. Vestibulum luctus nibh at lectus. Sed bibendum, nulla a faucibus semper, leo velit ultricies tellus, ac venenatis arcu wisi vel nisl. Vestibulum diam. Aliquam pellentesque, augue quis sagittis posuere, turpis lacus congue quam, in hendrerit risus eros eget felis.

Fusce mauris. Vestibulum luctus nibh at lectus. Sed bibendum, nulla a faucibus semper, leo velit ultricies tellus, ac venenatis arcu wisi vel nisl. Vestibulum diam. Aliquam pellentesque, augue quis sagittis posuere, turpis lacus congue quam, in hendrerit risus eros eget felis.\footnote{Vestibulum luctus nibh at lectus. Sed bibendum, nulla a faucibus semper, leo velit ultricies tellus, ac venenatis arcu wisi vel nisl.}

\section{Fundaciones}

Fusce mauris. Vestibulum luctus nibh at lectus. Sed bibendum, nulla a faucibus semper, leo velit ultricies tellus, ac venenatis arcu wisi vel nisl. Vestibulum diam. Aliquam pellentesque, augue quis sagittis posuere, turpis lacus congue quam, in hendrerit risus eros eget felis.

Fusce mauris. Vestibulum luctus nibh at lectus. Sed bibendum, nulla a faucibus semper, leo velit ultricies tellus, ac venenatis arcu wisi vel nisl. Vestibulum diam. Aliquam pellentesque, augue quis sagittis posuere, turpis lacus congue quam, in hendrerit risus eros eget felis.\footnote{Vestibulum luctus nibh at lectus. Sed bibendum, nulla a faucibus semper, leo velit ultricies tellus, ac venenatis arcu wisi vel nisl.}

\begin{quote}
	\enquote{Suspendisse vel felis. Ut lorem lorem, interdum eu, tincidunt sit amet, laoreet vitae, arcu. Aenean faucibus pede eu ante. Praesent enim elit, rutrum at, molestie non, nonummy vel, nisl} \parencite{@3189-QUIROGA2017}.
\end{quote}

Fusce mauris. Vestibulum luctus nibh at lectus. Sed bibendum, nulla a faucibus semper, leo velit ultricies tellus, ac venenatis arcu wisi vel nisl. Vestibulum diam. Aliquam pellentesque, augue quis sagittis posuere, turpis lacus congue quam, in hendrerit risus eros eget felis.

%cap2
\chapter{Metamorfosis}

Quisque ullamcorper placerat ipsum. Cras nibh. Morbi vel justo vitae lacus tincidunt ultrices. Lorem ipsum dolor sit amet, consectetuer adipiscing elit. In hac habitasse platea dictumst. Integer tempus convallis augue. Etiam facilisis. Nunc elementum fermentum wisi. Aenean placerat. Ut imperdiet, enim sed gravida sollicitudin, felis odio placerat quam, ac pulvinar elit purus eget enim. Nunc vitae tortor. Proin tempus nibh sit amet nisl. Vivamus quis tortor vitae risus porta vehicula.

Fusce mauris. Vestibulum luctus nibh at lectus. Sed bibendum, nulla a faucibus semper, leo velit ultricies tellus, ac venenatis arcu wisi vel nisl. Vestibulum diam. Aliquam pellentesque, augue quis sagittis posuere, turpis lacus congue quam, in hendrerit risus eros eget felis. Maecenas eget erat in sapien mattis porttitor. Vestibulum porttitor.

\section{La batalla de las primeras personas}

Quisque ullamcorper placerat ipsum. Cras nibh. Morbi vel justo vitae lacus tincidunt ultrices. Lorem ipsum dolor sit amet, consectetuer adipiscing elit. In hac habitasse platea dictumst. Integer tempus convallis augue. Etiam facilisis. Nunc elementum fermentum wisi. Aenean placerat. Ut imperdiet, enim sed gravida sollicitudin, felis odio placerat quam, ac pulvinar elit purus eget enim. Nunc vitae tortor. Proin tempus nibh sit amet nisl. Vivamus quis tortor vitae risus porta vehicula.

Fusce mauris. Vestibulum luctus nibh at lectus. Sed bibendum, nulla a faucibus semper, leo velit ultricies tellus, ac venenatis arcu wisi vel nisl. Vestibulum diam. Aliquam pellentesque, augue quis sagittis posuere, turpis lacus congue quam, in hendrerit risus eros eget felis. Maecenas eget erat in sapien mattis porttitor. Vestibulum porttitor.

\section{Los peligros del olvido de los orígenes}

Quisque ullamcorper placerat ipsum. Cras nibh. Morbi vel justo vitae lacus tincidunt ultrices. Lorem ipsum dolor sit amet, consectetuer adipiscing elit. In hac habitasse platea dictumst. Integer tempus convallis augue. Etiam facilisis. Nunc elementum fermentum wisi. Aenean placerat. Ut imperdiet, enim sed gravida sollicitudin, felis odio placerat quam, ac pulvinar elit purus eget enim. Nunc vitae tortor. Proin tempus nibh sit amet nisl. Vivamus quis tortor vitae risus porta vehicula.

Fusce mauris. Vestibulum luctus nibh at lectus. Sed bibendum, nulla a faucibus semper, leo velit ultricies tellus, ac venenatis arcu wisi vel nisl. Vestibulum diam. Aliquam pellentesque, augue quis sagittis posuere, turpis lacus congue quam, in hendrerit risus eros eget felis. Maecenas eget erat in sapien mattis porttitor. Vestibulum porttitor.

%cap3
\chapter{La victoria, la muerte y la representación}

Quisque ullamcorper placerat ipsum. Cras nibh. Morbi vel justo vitae lacus tincidunt ultrices. Lorem ipsum dolor sit amet, consectetuer adipiscing elit. In hac habitasse platea dictumst. Integer tempus convallis augue. Etiam facilisis. Nunc elementum fermentum wisi. Aenean placerat. Ut imperdiet, enim sed gravida sollicitudin, felis odio placerat quam, ac pulvinar elit purus eget enim. Nunc vitae tortor. Proin tempus nibh sit amet nisl. Vivamus quis tortor vitae risus porta vehicula.

\section{Promulgación de la ley n.° 13.010: discurso de Eva Duarte (1947)}

Quisque ullamcorper placerat ipsum. Cras nibh. Morbi vel justo vitae lacus tincidunt ultrices. Lorem ipsum dolor sit amet, consectetuer adipiscing elit. In hac habitasse platea dictumst. Integer tempus convallis augue. Etiam facilisis. Nunc elementum fermentum wisi. Aenean placerat. Ut imperdiet, enim sed gravida sollicitudin, felis odio placerat quam, ac pulvinar elit purus eget enim. Nunc vitae tortor. Proin tempus nibh sit amet nisl. Vivamus quis tortor vitae risus porta vehicula.

Fusce mauris. Vestibulum luctus nibh at lectus. Sed bibendum, nulla a faucibus semper, leo velit ultricies tellus, ac venenatis arcu wisi vel nisl. Vestibulum diam. Aliquam pellentesque, augue quis sagittis posuere, turpis lacus congue quam, in hendrerit risus eros eget felis. Maecenas eget erat in sapien mattis porttitor. Vestibulum porttitor.

\section{Muerte y desaparición del cuerpo de Eva Duarte}

Quisque ullamcorper placerat ipsum. Cras nibh. Morbi vel justo vitae lacus tincidunt ultrices. Lorem ipsum dolor sit amet, consectetuer adipiscing elit. In hac habitasse platea dictumst. Integer tempus convallis augue. Etiam facilisis. Nunc elementum fermentum wisi. Aenean placerat. Ut imperdiet, enim sed gravida sollicitudin, felis odio placerat quam, ac pulvinar elit purus eget enim. Nunc vitae tortor. Proin tempus nibh sit amet nisl. Vivamus quis tortor vitae risus porta vehicula.

Fusce mauris. Vestibulum luctus nibh at lectus. Sed bibendum, nulla a faucibus semper, leo velit ultricies tellus, ac venenatis arcu wisi vel nisl. Vestibulum diam. Aliquam pellentesque, augue quis sagittis posuere, turpis lacus congue quam, in hendrerit risus eros eget felis. Maecenas eget erat in sapien mattis porttitor. Vestibulum porttitor.

%cap4
\chapter{Viajes en el tiempo}

Quisque ullamcorper placerat ipsum. Cras nibh. Morbi vel justo vitae lacus tincidunt ultrices. Lorem ipsum dolor sit amet, consectetuer adipiscing elit. In hac habitasse platea dictumst. Integer tempus convallis augue. Etiam facilisis. Nunc elementum fermentum wisi. Aenean placerat. Ut imperdiet, enim sed gravida sollicitudin, felis odio placerat quam, ac pulvinar elit purus eget enim. Nunc vitae tortor. Proin tempus nibh sit amet nisl. Vivamus quis tortor vitae risus porta vehicula.

Fusce mauris. Vestibulum luctus nibh at lectus. Sed bibendum, nulla a faucibus semper, leo velit ultricies tellus, ac venenatis arcu wisi vel nisl. Vestibulum diam. Aliquam pellentesque, augue quis sagittis posuere, turpis lacus congue quam, in hendrerit risus eros eget felis. Maecenas eget erat in sapien mattis porttitor. Vestibulum porttitor.

\section{Argentinización de la galaxia: ciencia ficción}

Quisque ullamcorper placerat ipsum. Cras nibh. Morbi vel justo vitae lacus tincidunt ultrices. Lorem ipsum dolor sit amet, consectetuer adipiscing elit. In hac habitasse platea dictumst. Integer tempus convallis augue. Etiam facilisis. Nunc elementum fermentum wisi. Aenean placerat. Ut imperdiet, enim sed gravida sollicitudin, felis odio placerat quam, ac pulvinar elit purus eget enim. Nunc vitae tortor. Proin tempus nibh sit amet nisl. Vivamus quis tortor vitae risus porta vehicula.

Fusce mauris. Vestibulum luctus nibh at lectus. Sed bibendum, nulla a faucibus semper, leo velit ultricies tellus, ac venenatis arcu wisi vel nisl. Vestibulum diam. Aliquam pellentesque, augue quis sagittis posuere, turpis lacus congue quam, in hendrerit risus eros eget felis. Maecenas eget erat in sapien mattis porttitor. Vestibulum porttitor.

\section{Autobiografía: los viajes en el tiempo del realismo}

Quisque ullamcorper placerat ipsum. Cras nibh. Morbi vel justo vitae lacus tincidunt ultrices. Lorem ipsum dolor sit amet, consectetuer adipiscing elit. In hac habitasse platea dictumst. Integer tempus convallis augue. Etiam facilisis. Nunc elementum fermentum wisi. Aenean placerat. Ut imperdiet, enim sed gravida sollicitudin, felis odio placerat quam, ac pulvinar elit purus eget enim. Nunc vitae tortor. Proin tempus nibh sit amet nisl. Vivamus quis tortor vitae risus porta vehicula.

Fusce mauris. Vestibulum luctus nibh at lectus. Sed bibendum, nulla a faucibus semper, leo velit ultricies tellus, ac venenatis arcu wisi vel nisl. Vestibulum diam. Aliquam pellentesque, augue quis sagittis posuere, turpis lacus congue quam, in hendrerit risus eros eget felis. Maecenas eget erat in sapien mattis porttitor. Vestibulum porttitor.

\section{Ampliación de los límites y las fronteras de los géneros}

Quisque ullamcorper placerat ipsum. Cras nibh. Morbi vel justo vitae lacus tincidunt ultrices. Lorem ipsum dolor sit amet, consectetuer adipiscing elit. In hac habitasse platea dictumst. Integer tempus convallis augue. Etiam facilisis. Nunc elementum fermentum wisi. Aenean placerat. Ut imperdiet, enim sed gravida sollicitudin, felis odio placerat quam, ac pulvinar elit purus eget enim. Nunc vitae tortor. Proin tempus nibh sit amet nisl. Vivamus quis tortor vitae risus porta vehicula.

Fusce mauris. Vestibulum luctus nibh at lectus. Sed bibendum, nulla a faucibus semper, leo velit ultricies tellus, ac venenatis arcu wisi vel nisl. Vestibulum diam. Aliquam pellentesque, augue quis sagittis posuere, turpis lacus congue quam, in hendrerit risus eros eget felis. Maecenas eget erat in sapien mattis porttitor. Vestibulum porttitor.

\section{La literatura de terror: ¿para qué sirve narrar?}

Quisque ullamcorper placerat ipsum. Cras nibh. Morbi vel justo vitae lacus tincidunt ultrices. Lorem ipsum dolor sit amet, consectetuer adipiscing elit. In hac habitasse platea dictumst. Integer tempus convallis augue. Etiam facilisis. Nunc elementum fermentum wisi. Aenean placerat. Ut imperdiet, enim sed gravida sollicitudin, felis odio placerat quam, ac pulvinar elit purus eget enim. Nunc vitae tortor. Proin tempus nibh sit amet nisl. Vivamus quis tortor vitae risus porta vehicula.

Fusce mauris. Vestibulum luctus nibh at lectus. Sed bibendum, nulla a faucibus semper, leo velit ultricies tellus, ac venenatis arcu wisi vel nisl. Vestibulum diam. Aliquam pellentesque, augue quis sagittis posuere, turpis lacus congue quam, in hendrerit risus eros eget felis. Maecenas eget erat in sapien mattis porttitor. Vestibulum porttitor.

\section{Literatura contemporánea}

Quisque ullamcorper placerat ipsum. Cras nibh. Morbi vel justo vitae lacus tincidunt ultrices. Lorem ipsum dolor sit amet, consectetuer adipiscing elit. In hac habitasse platea dictumst. Integer tempus convallis augue. Etiam facilisis. Nunc elementum fermentum wisi. Aenean placerat. Ut imperdiet, enim sed gravida sollicitudin, felis odio placerat quam, ac pulvinar elit purus eget enim. Nunc vitae tortor. Proin tempus nibh sit amet nisl. Vivamus quis tortor vitae risus porta vehicula.

Fusce mauris. Vestibulum luctus nibh at lectus. Sed bibendum, nulla a faucibus semper, leo velit ultricies tellus, ac venenatis arcu wisi vel nisl. Vestibulum diam. Aliquam pellentesque, augue quis sagittis posuere, turpis lacus congue quam, in hendrerit risus eros eget felis. Maecenas eget erat in sapien mattis porttitor. Vestibulum porttitor.

%La secuencia de Fibonacci es una serie matemática en la cual cada número es la suma de los dos anteriores, comenzando generalmente con 0 y 1. La fórmula general para la secuencia de Fibonacci se puede expresar matemáticamente de la siguiente manera:
%
%$$F(n) = F(n-1) + F(n-2)$$
%
%\ifPDF
%\froufrou
%\else
%	\ifBLACKPDF
%	\froufrou
%	\else
%		\ifODT
%		\begin{center} * * * \end{center}
%		\fi
%	\fi
%\fi
%
%\section{SecciónAA}
%
%Esta obra colectiva da continuidad a uno de los programas centrales impulsados por el \gls{@glo201-ceheal} desde su creación en 2019: el estudio de las ideas y del pensamiento económico en su vínculo con la implementación de políticas económicas\index[concepto]{Políticas económicas para la sociedad} \parencite{@940-SHUMWAY1999}.
%
%\chapter{MainmatterB}
%
%\section{SecciónB}
%
%Esta obra colectiva da continuidad a uno de los programas centrales impulsados por el \gls{@glo201-ceheal} desde su creación en 2019: el estudio de las ideas y del pensamiento económico en su vínculo con la implementación de políticas económicas.
%
%\ifPDF%
%\begin{figure}[!ht]
%\centering
%\includegraphics[width=\textwidth]{./media/imagen1.jpg}
%\caption{Este es el epígrafe de la figura a color, véase \textcite{@3159-FUNES2006}.}
%\end{figure}
%\else
%	\ifBLACKPDF%
%	\begin{figure}[!ht]
%	\centering
%	\includegraphics[width=\textwidth]{./media/bn-imagen1.png}
%	\caption{Este es el epígrafe de la figura en escala de grises.}
%	\end{figure}
%	\else
%		\ifODT%
%		\begin{figure}[!ht]
%		\centering
%		\includegraphics[width=\textwidth]{./media/imagen1.jpg}
%		\caption{Este es el epígrafe de la figura a color.}
%		\end{figure}
%		\else
%			\ifEPUB
%			\begin{figure}[!ht]
%			\centering
%			\includegraphics[width=\textwidth]{./media/bn-imagen1.png}
%			\caption{Este es el epígrafe de la figura a color.}
%			\end{figure}
%			\fi
%		\fi
%	\fi
%\fi
%
%\section{SecciónBB}
%
%Esta obra colectiva da continuidad a uno de los programas centrales impulsados por el \gls{@glo201-ceheal} desde su creación en 2019: el estudio de las ideas y del pensamiento económico en su vínculo con la implementación de políticas económicas.\footnote{Nota a pie en una sección del segundo capítulo.}

\backmatter

\ifPDF
\printnoidxglossary[type=\acronymtype,title={Índice de siglas}]
\printnoidxglossary[title={Glosario de términos}]
\printbibliography[heading=none,heading=bibintoc]
\printindex[names]
\printindex[concepto]
\printindex[onomastico]
\Author{Índice de autoras y autores}
	\else
	\ifBLACKPDF
		\printnoidxglossary[type=\acronymtype,title={Índice de siglas}]
		\printnoidxglossary[title={Glosario de términos}]
		\printbibliography[heading=none,heading=bibintoc]
	\printindex[names]
	\printindex[concepto]
	\printindex[onomastico]
	\Author{Índice de autoras y autores}
		\else
	 	\ifEPUB
	 	\begingroup
	 	\parindent 0pt
	 	\parskip 2ex
	 	\def\enotesize{\normalsize}
	 	\theendnotes
	 	\endgroup
	 	\printnoidxglossary[type=\acronymtype,title={Índice de siglas}]
	 	\printnoidxglossary[title={Glosario de términos}]
	 	\chapter{Referencias bibliográficas}
	 	\printbibliography[heading=none]
%	 	\printindex[names]
%	 	\printindex[concepto]
%	 	\printindex[onomastico]
	 		\else
	 		\ifHTML
	 		\ForceHTMLPage
	 		\printnoidxglossary[title={Índice de siglas},type=\acronymtype]
	 		\ForceHTMLPage
	 		\printnoidxglossary[title={Glosario de términos}]
	 		\ForceHTMLPage
	 		\printbibliography[heading=bibintoc]
	 			\else
	 			\ifODT
	 			\printnoidxglossary[type=\acronymtype,title={Índice de siglas}]
	 			\printnoidxglossary[title={Glosario de términos}]
	 			\printbibliography[heading=none,heading=bibintoc]
	 			\fi
	 		\fi
		\fi
	\fi
\fi



\chapter{Colofón}

La composición tipográfica de este libro se realizó utilizando \href{https://github.com/albertomoyano/gbtexpublisher}{gbTeXpublisher.}

Las familias tipográficas utilizadas dentro del libro son: IBM Plex, una superfamilia de tipografía abierta, diseñada y desarrollada conceptualmente por Mike Abbink en IBM con colaboración de Bold Monday y Libertinus, bifurcación de la fuente Linux Libertine, diseñada para el texto del cuerpo y la lectura extendida.

\ifPDF
\newpage
\thispagestyle{empty}
{\textcolor{white}{.}}
	\else
	\ifBLACKPDF
	\newpage
	\thispagestyle{empty}
	{\textcolor{white}{.}}
	\fi
\fi

\end{document}



